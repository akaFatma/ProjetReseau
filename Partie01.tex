\section{Mise en place de la topologie} 
\vspace{0.3cm}

\subsection{Confirmation de la configuration en utilisant ping}
\begin{figure}[h]
    \centering
    \includegraphics[width=1\textwidth]{./images/topology.png} 
    \caption{Topologie du test 1}
    \label{fig:exemple}
\end{figure}

\subsubsection{Changement du masque de sous-réseau}
\vspace{0.3cm}
Initialement, le masque de sous-réseau était défini à /8, ce qui signifie que les 8 premiers bits de l'adresse IP représentaient la partie réseau, et les 24 bits restants étaient réservés pour les hôtes.
Cependant, on doit modifier ce masque de sous-réseau de /8 à /16. Cela signifie que les 16 premiers bits définissent désormais la partie réseau de l’adresse IP, tandis que les 16 bits restants sont utilisés pour les hôtes.

\begin{figure}[h]
    \centering
    \includegraphics[width=1\textwidth]{./images/IpBaseSetup.png} 
    \caption{Changement du masque de sous-réseau}
    \label{fig:exemple}
\end{figure}


\subsubsection{Attribution des Adresses IP aux Hôtes}
\begin{itemize}
    \item h1 : 10.25.0.26
    \item h2 : 10.25.0.27
\end{itemize}


\subsubsection{Ping}
Lorsque la commande ping est lancée, h2 envoie un paquet ICMP Echo Request à l'adresse IP de h1. Ce paquet contient 64 bytes de données par défaut, ainsi qu'un numéro de séquence pour suivre les paquets envoyés (icmp\_seq).

h1 reçoit ce paquet ICMP Echo Request et le traite. En réponse, il génère un paquet ICMP Echo Reply, qui contient les mêmes données que celles envoyées par h2.

Ce paquet Echo Reply est renvoyé à l'appareil source (h2). Lorsque h2 reçoit la réponse, il mesure le temps que le paquet a mis pour faire l'aller-retour, appelé Round Trip Time (RTT).

Le terminal affiche ainsi le TTL (Time to Live), qui mesure le nombre de routeurs ou de sauts traversés par le paquet avant d'atteindre sa destination.
\begin{itemize}
    \item \textbf{64 bytes} : La taille du paquet ICMP envoyé et reçu (64 octets par défaut).
    \item \textbf{from 10.25.0.26} : L'adresse IP de la machine qui a répondu à la requête ICMP (ici, h1).
    \item \textbf{icmp\_seq=i} : Le numéro de séquence du paquet envoyé, permettant de vérifier l'ordre et la réception des paquets.
    \item \textbf{ttl=64} : Le Time to Live du paquet, qui est initialement défini par l'appareil source est décrémenté de 1 à chaque routeur traversé.
    \item \textbf{time=0.072 ms} : Le temps que le paquet a mis pour faire l’aller-retour, en millisecondes, indiquant la latence ou délai entre les deux appareils.
\end{itemize}
\begin{figure}[h]
    \centering
    \includegraphics[width=0.6\textwidth]{./images/ping.png} 
    \caption{Commande ping}
    \label{fig:exemple}
\end{figure}
\newpage
\section{Tests}
\subsection{T1.1 : Varier le débit du lien entre H1 et H2 }
\vspace{0.5cm}
\hspace{0.5cm}\textbf{1:} `ip link show` et `tc qdisc show dev h1-eth0`
\begin{center}
    \includegraphics[width=1\textwidth]{./images/ShowDefaultQdisc.png}
\end{center}
Le résultat de ip link show indiquera si l'interface h1-eth0 est active et prête à envoyer des données, tandis que le résultat de tc qdisc show montrera les règles en place pour le contrôle du trafic, y compris les limites de bande passante et la gestion de la latence. 

\vspace{1cm}

\textbf{2:} `tc qdisc add dev h1-eth0 root handle 1:0 tbf rate 2.5gbit burst 32kbit latency 400ms (resp h2-eth0)`
\begin{center}
    \includegraphics[width=1\textwidth]{./images/1smaller.png}
\end{center}
Les deux interfaces h1-eth0 et h2-eth0 sont configurées pour limiter le débit à 2,5 Gbps.(E1)

\vspace{1cm}
\textbf{3:} `iperf3 -s pour h1 , iperf3 -c ip-h1`
\begin{center}
    \includegraphics[width=1\textwidth]{./images/2smaller.png}
\end{center}
Pour h1 : iperf3 -s lance iperf3 en mode serveur , elle accepte la connexion au client h2\\
pour h2 : iperf3 -c 10.25.0.26 lance iperf3 en mode client et connecte au seurveur h1
\\
\\Les débits moyens de 493 Mbits/sec et 494 Mbits/sec sont observés respectivement pour le serveur et le client. Cela montre que les deux hôtes peuvent échanger des données à une vitesse proche de 500 Mbits/sec.
\\
\\Bien que les commandes tc qdisc add initiales aient fixé un débit maximum de 2,5 Gbits/sec, les résultats montrent des débits bien inférieurs à cette limite (environ 493-494 Mbits/sec).
\\ 
\\les memes tests sont fait pour une valeur moyenne (DL * 0.05) et une valeur élevée (DL par defaut), on preésente les captures : 

\vspace{1cm}
\textbf{4:} `tc qdisc add dev h1-eth0 root handle 1:0 tbf rate 125Mbit burst 32kbit latency 400ms (resp h2-eth0)`
\begin{center}
    \includegraphics[width=1\textwidth]{./images/ValuerMoyenneCommande.png} \\
    \includegraphics[width=1\textwidth]{./images/ValeurMoyenneResultat.png}
\end{center}
\vspace{1cm}
\hspace{0.5cm}\textbf{5:} `tc qdisc add dev h1-eth0 root handle 1:0 tbf rate 625Mbit burst 32kbit latency 400ms (resp h2-eth0)`
\begin{center}
    \includegraphics[width=1\textwidth]{./images/ValuerDefautCommande.png} 
    \includegraphics[width=1\textwidth]{./images/ValuerDefautResultat.png}
\end{center}
\textbf{Courbe T1.1:} 

Le graphe montre une relation entre le débit physique du lien (DL) et le débit applicatif mesuré lors des tests entre les machines h1 et h2.
\\
\\Le débit applicatif augmente de manière significative avec le débit physique , puis se stabilise, indiquant que d'autres facteurs, comme les limites du CPU ou l'outil iperf3 lui meme, empêchent une augmentation proportionnelle, malgré l'augmentation du débit physique.
 
\newpage
\subsection{T1.2 : Fixer la valeur du débit sur le lien entre H1 et le Switch à 1Gb/s et varier le débit sur H2.}
\vspace{0.5cm}
\begin{center}
    \includegraphics[width=1\textwidth]{./images/T1.2TopologyAndPing.png}
\end{center}
\vspace{1cm}
\textbf{1:} `tc qdisc add dev h2-eth0 root handle 1:0 tbf rate 2.5gbit burst 32kbit latency 400ms`
\begin{center}
    \includegraphics[width=1\textwidth]{./images/11.png}
\end{center}
hnaya commentaire\\
\\
\vspace{1cm}
\textbf{2:} `tc qdisc add dev h2-eth0 root handle 1:0 tbf rate 2.5gbit burst 32kbit latency 400ms`
\begin{center}
    \includegraphics[width=1\textwidth]{./images/13.png}
\end{center}
hnaya commentaire\\
\\
\vspace{1cm}
\textbf{3:} `tc qdisc add dev h2-eth0 root handle 1:0 tbf rate 2.5gbit burst 32kbit latency 400ms`
\begin{center}
    \includegraphics[width=1\textwidth]{./images/15.png}
\end{center}

\newpage
\textbf{Courbe T1.2:} 
\begin{center}
    \includegraphics[width=1\textwidth]{./images/CourbeT1.2.png} \\
\end{center}
hnaya commentaire\\
\\
\textbf{Comparaison entre les deux tests} 






\newpage
\subsection{Comparaison entre les résultats des tests T1.1 et T1.2}
\vspace{0.5cm}
\subsubsection{Résultats pour un Débit Physique de 125 Mbit/s}
\begin{figure}[H]
    \centering
    \includegraphics[width=1\textwidth]{./images/T1vsT2pour125.png}
    \caption{Courbe de Test1 VS Test2 pour debit physique = 125Mbit/s}
    \label{fig:exemple}
\end{figure}

\textbf{Analyse :}\\
\textbf{Comportement Initial :} T1.2 affiche un débit initial plus élevé (environ 145 Mbit/s) par rapport à T1.1, qui démarre autour de 125 Mbit/s.\\
\textbf{Stabilité et Fluctuations :} Les deux configurations atteignent un pic de débit entre 2 et 3 secondes,  mais ensuite, elles diminuent pour prendre des valeurs inférieurs à 125 Mbit/s\\
\textbf{Conclusion (125 Mbit/s) :}  Les deux configurations commencent de manière différente, mais après un certain temps, les deux configurations finissent par se rapprocher en termes de comportement, montrant une instabilité similaire.

\subsubsection{Résultats pour un Débit Physique de 625 Mbit/s}
\begin{figure}[H]
    \centering
    \includegraphics[width=1\textwidth]{./images/T1vsT2pour625.png}
    \caption{Courbe de Test1 VS Test2 pour debit physique = 625Mbit/s}
    \label{fig:exemple}
\end{figure}

\textbf{Analyse :}\\
\textbf{Comportement Initial :} T1.1 atteint rapidement environ 600 Mbit/s et reste stable, tandis que T1.2 montre des fluctuations initiales, atteignant environ 550 Mbit/s.\\
\textbf{Stabilité et Consistance :} T1.1 est stable et proche de la limite physique, tandis que T1.2 est instable, avec des variations importantes.\\
\textbf{Conclusion (625 Mbit/s) :} Le lien direct (T1.1) offre une performance supérieure, avec un débit plus stable.

\subsubsection{Interprétation des Résultats pour un Débit Physique de 2,5 Gbit/s}
\begin{figure}[H]
    \centering
    \includegraphics[width=1\textwidth]{./images/T1vsT2pour2500.png}
    \caption{Courbe de Test1 VS Test2 pour debit physique = 2.5Gbit/s}
    \label{fig:exemple}
\end{figure}

\textbf{Analyse :}\\
\textbf{Performance du Lien Direct (T1.1) :}Le débit mesuré est d'environ 2,4 Gbit/s, indiquant une communication efficace et proche de la capacité maximale du lien.\\
\textbf{Performance avec le Switch (T1.2) :} Le débit fluctue autour de 600 Mbit/s, en raison de la limitation imposée par le débit physique entre le switch et l'hôte, d'éventuels goulots d'étranglement et des processus de traitement dans le switch\\
\textbf{Conclusion (2,5 Gbit/s) :} Le lien direct est plus performant à des débits élevés, étant donné que l'ajout d'un switch entraîne une dégradation notable du débit.
\newpage
\subsubsection{Interprétation Générale: }
\textbf{Impact de la Topologie du Réseau :}\\
L’ajout d’un switch avec une limitation de débit modifie fondamentalement la capacité du réseau à transmettre un débit élevé.L’absence de goulot d’étranglement dans le Test 1 permet d’atteindre un débit applicatif optimal, alors que la présence du switch dans le Test 2 empêche l’hôte h2 de bénéficier d’une augmentation du débit physique.

\textbf{Lien Direct (T1.1) :} Offre une meilleure performance, minimisant les latences et maximisant l'utilisation du débit physique.

\textbf{Switch (T1.2) :} Introduit une surcharge, réduisant la performance, surtout à des débits élevés.\\

Pour des applications nécessitant un débit élevé et constant, une connexion directe est préférable. L'utilisation d'un switch est recommandée pour des scénarios nécessitant flexibilité et expansion du réseau, bien qu'elle puisse entraîner des baisses de performance. En résumé, le choix de la topologie doit être basé sur les exigences de l'application, en équilibrant performance et flexibilité.
\section {Conclusion: }
Les résultats des tests réalisés mettent en évidence l'impact significatif de la configuration du réseau et des équipements intermédiaires sur le débit applicatif. Lorsqu'il n'y a pas de limitations causées par des éléments intermédiaires tels que des switches, l'augmentation du débit physique se traduit directement par une amélioration du débit applicatif. Cela suggère que, dans des conditions idéales, le débit applicatif peut évoluer de manière linéaire avec le débit physique, permettant ainsi d’optimiser les performances du réseau. Cependant, l'introduction de restrictions intermediaires, notamment par la présence d'un switch avec un débit limité, peut entraîner une saturation du débit applicatif. Dans ce cas, bien que le débit physique sur d'autres segments du réseau soit augmenté, cela n’a que peu ou pas d'effet sur le débit applicatif, qui atteint rapidement un plafond. Ce phénomène souligne l'importance de la gestion adéquate de la capacité des équipements intermédiaires, afin de maximiser l'efficacité du réseau et d’éviter toute dégradation des performances due à des limitations imprévues.

