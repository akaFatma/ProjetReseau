\section*{Mise en place de la topologie} 
\vspace{0.3cm}

\subsection{Confirmation de la configuration en utilisant ping}
\includegraphics[width=1\textwidth]{./images/topology.png} \\
\subsubsection{Changement du masque de sous-réseau}
\vspace{0.3cm}
Initialement, le masque de sous-réseau était défini à /8, ce qui signifie que les 8 premiers bits de l'adresse IP représentaient la partie réseau, et les 24 bits restants étaient réservés pour les hôtes.
Cependant, on doit modifier ce masque de sous-réseau de /8 à /16. Cela signifie que les 16 premiers bits définissent désormais la partie réseau de l’adresse IP, tandis que les 16 bits restants sont utilisés pour les hôtes.\\

\includegraphics[width=1\textwidth]{./images/IpBaseSetup.png} \\


\subsubsection{Attribution des Adresses IP aux Hôtes}
\begin{itemize}
    \item h1 : 10.25.0.26
    \item h2 : 10.25.0.27
\end{itemize}


\subsubsection{Ping}
Lorsque la commande ping est lancée, h2 envoie un paquet ICMP Echo Request à l'adresse IP de h1. Ce paquet contient 64 bytes de données par défaut, ainsi qu'un numéro de séquence pour suivre les paquets envoyés (icmp\_seq).

h1 reçoit ce paquet ICMP Echo Request et le traite. En réponse, il génère un paquet ICMP Echo Reply, qui contient les mêmes données que celles envoyées par h2.

Ce paquet Echo Reply est renvoyé à l'appareil source (h2). Lorsque h2 reçoit la réponse, il mesure le temps que le paquet a mis pour faire l'aller-retour, appelé Round Trip Time (RTT).

Le terminal affiche ainsi le TTL (Time to Live), qui mesure le nombre de routeurs ou de sauts traversés par le paquet avant d'atteindre sa destination.

\begin{itemize}
    \item \textbf{64 bytes} : La taille du paquet ICMP envoyé et reçu (64 octets par défaut).
    \item \textbf{from 10.25.0.26} : L'adresse IP de la machine qui a répondu à la requête ICMP (ici, h1).
    \item \textbf{icmp\_seq=i} : Le numéro de séquence du paquet envoyé, permettant de vérifier l'ordre et la réception des paquets.
    \item \textbf{ttl=64} : Le Time to Live du paquet, qui est initialement défini par l'appareil source est décrémenté de 1 à chaque routeur traversé.
    \item \textbf{time=0.072 ms} : Le temps que le paquet a mis pour faire l’aller-retour, en millisecondes, indiquant la latence ou délai entre les deux appareils.
\end{itemize}
\begin{center}
    \includegraphics[width=1\textwidth]{./images/TopologiePing.png}
\end{center}

\subsection{Tests}
\subsubsection{T1.1 : Varier le débit du lien entre H1 et H2 ( mettre la même valeur sur les deux machines ).}
\vspace{0.5cm}
\hspace{0.5cm}\textbf{1:} `ip link show` et `tc qdisc show dev h1-eth0`
\begin{center}
    \includegraphics[width=1\textwidth]{./images/ShowDefaultQdisc.png}
\end{center}
Le résultat de ip link show indiquera si l'interface h1-eth0 est active et prête à envoyer des données, tandis que le résultat de tc qdisc show montrera les règles en place pour le contrôle du trafic, y compris les limites de bande passante et la gestion de la latence. 

\vspace{1cm}

\textbf{2:} `tc qdisc add dev h1-eth0 root handle 1:0 tbf rate 2.5gbit burst 32kbit latency 400ms (resp h2-eth0)`
\begin{center}
    \includegraphics[width=1\textwidth]{./images/1smaller.png}
\end{center}
Les deux interfaces h1-eth0 et h2-eth0 sont configurées pour limiter le débit à 2,5 Gbps.(E1)

\vspace{1cm}
\textbf{3:} `iperf3 -s pour h1 , iperf3 -c ip-h1`
\begin{center}
    \includegraphics[width=1\textwidth]{./images/2smaller.png}
\end{center}
Pour h1 : iperf3 -s lance iperf3 en mode serveur , elle accepte la connexion au client h2\\
pour h2 : iperf3 -c 10.25.0.26 lance iperf3 en mode client et connecte au seurveur h1
\\
\\Les débits moyens de 493 Mbits/sec et 494 Mbits/sec sont observés respectivement pour le serveur et le client. Cela montre que les deux hôtes peuvent échanger des données à une vitesse proche de 500 Mbits/sec.
\\
\\Bien que les commandes tc qdisc add initiales aient fixé un débit maximum de 2,5 Gbits/sec, les résultats montrent des débits bien inférieurs à cette limite (environ 493-494 Mbits/sec).
\\ 
\\les memes tests sont fait pour une valeur moyenne (DL * 0.05) et une valeur élevée (DL par defaut), on preésente les captures : 

\vspace{1cm}
\textbf{4:} `tc qdisc add dev h1-eth0 root handle 1:0 tbf rate 125Mbit burst 32kbit latency 400ms (resp h2-eth0)`
\begin{center}
    \includegraphics[width=1\textwidth]{./images/ValuerMoyenneCommande.png} \\
    \includegraphics[width=1\textwidth]{./images/ValeurMoyenneResultat.png}
\end{center}
\vspace{1cm}
\hspace{0.5cm}\textbf{5:} `tc qdisc add dev h1-eth0 root handle 1:0 tbf rate 625Mbit burst 32kbit latency 400ms (resp h2-eth0)`
\begin{center}
    \includegraphics[width=1\textwidth]{./images/ValuerDefautCommande.png} 
    \includegraphics[width=1\textwidth]{./images/ValuerDefautResultat.png}
\end{center}
\textbf{Courbe T1.1:} 

Le graphe montre une relation entre le débit physique du lien (DL) et le débit applicatif mesuré lors des tests entre les machines h1 et h2.
\\
\\Le débit applicatif augmente de manière significative avec le débit physique , puis se stabilise, indiquant que d'autres facteurs, comme les limites du CPU ou l'outil iperf3 lui meme, empêchent une augmentation proportionnelle, malgré l'augmentation du débit physique.
 
\subsubsection{T1.2 : Fixer la valeurdu débit sur le lien entre H1 et le Switch à 1Gb/s et varier le débit sur H2.}
\vspace{0.5cm}
\begin{center}
    \includegraphics[width=1\textwidth]{./images/T1.2TopologyAndPing.png}
\end{center}
\vspace{1cm}
\textbf{1:} `tc qdisc add dev h2-eth0 root handle 1:0 tbf rate 2.5gbit burst 32kbit latency 400ms`
\begin{center}
    \includegraphics[width=1\textwidth]{./images/11.png}
\end{center}
hnaya commentaire\\
\\
\vspace{1cm}
\textbf{2:} `tc qdisc add dev h2-eth0 root handle 1:0 tbf rate 2.5gbit burst 32kbit latency 400ms`
\begin{center}
    \includegraphics[width=1\textwidth]{./images/13.png}
\end{center}
hnaya commentaire\\
\\
\vspace{1cm}
\textbf{3:} `tc qdisc add dev h2-eth0 root handle 1:0 tbf rate 2.5gbit burst 32kbit latency 400ms`
\begin{center}
    \includegraphics[width=1\textwidth]{./images/15.png}
\end{center}

\newpage
\textbf{Courbe T1.2:} 
\begin{center}
    \includegraphics[width=1\textwidth]{./images/CourbeT1.2.png} \\
\end{center}
hnaya commentaire\\
\\
\textbf{Comparaison entre les deux tests} 





